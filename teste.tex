% Options for packages loaded elsewhere
\PassOptionsToPackage{unicode}{hyperref}
\PassOptionsToPackage{hyphens}{url}
\PassOptionsToPackage{dvipsnames,svgnames,x11names}{xcolor}
%
\documentclass[
  letterpaper,
  DIV=11,
  numbers=noendperiod]{scrartcl}

\usepackage{amsmath,amssymb}
\usepackage{iftex}
\ifPDFTeX
  \usepackage[T1]{fontenc}
  \usepackage[utf8]{inputenc}
  \usepackage{textcomp} % provide euro and other symbols
\else % if luatex or xetex
  \usepackage{unicode-math}
  \defaultfontfeatures{Scale=MatchLowercase}
  \defaultfontfeatures[\rmfamily]{Ligatures=TeX,Scale=1}
\fi
\usepackage{lmodern}
\ifPDFTeX\else  
    % xetex/luatex font selection
\fi
% Use upquote if available, for straight quotes in verbatim environments
\IfFileExists{upquote.sty}{\usepackage{upquote}}{}
\IfFileExists{microtype.sty}{% use microtype if available
  \usepackage[]{microtype}
  \UseMicrotypeSet[protrusion]{basicmath} % disable protrusion for tt fonts
}{}
\makeatletter
\@ifundefined{KOMAClassName}{% if non-KOMA class
  \IfFileExists{parskip.sty}{%
    \usepackage{parskip}
  }{% else
    \setlength{\parindent}{0pt}
    \setlength{\parskip}{6pt plus 2pt minus 1pt}}
}{% if KOMA class
  \KOMAoptions{parskip=half}}
\makeatother
\usepackage{xcolor}
\setlength{\emergencystretch}{3em} % prevent overfull lines
\setcounter{secnumdepth}{-\maxdimen} % remove section numbering
% Make \paragraph and \subparagraph free-standing
\makeatletter
\ifx\paragraph\undefined\else
  \let\oldparagraph\paragraph
  \renewcommand{\paragraph}{
    \@ifstar
      \xxxParagraphStar
      \xxxParagraphNoStar
  }
  \newcommand{\xxxParagraphStar}[1]{\oldparagraph*{#1}\mbox{}}
  \newcommand{\xxxParagraphNoStar}[1]{\oldparagraph{#1}\mbox{}}
\fi
\ifx\subparagraph\undefined\else
  \let\oldsubparagraph\subparagraph
  \renewcommand{\subparagraph}{
    \@ifstar
      \xxxSubParagraphStar
      \xxxSubParagraphNoStar
  }
  \newcommand{\xxxSubParagraphStar}[1]{\oldsubparagraph*{#1}\mbox{}}
  \newcommand{\xxxSubParagraphNoStar}[1]{\oldsubparagraph{#1}\mbox{}}
\fi
\makeatother


\providecommand{\tightlist}{%
  \setlength{\itemsep}{0pt}\setlength{\parskip}{0pt}}\usepackage{longtable,booktabs,array}
\usepackage{calc} % for calculating minipage widths
% Correct order of tables after \paragraph or \subparagraph
\usepackage{etoolbox}
\makeatletter
\patchcmd\longtable{\par}{\if@noskipsec\mbox{}\fi\par}{}{}
\makeatother
% Allow footnotes in longtable head/foot
\IfFileExists{footnotehyper.sty}{\usepackage{footnotehyper}}{\usepackage{footnote}}
\makesavenoteenv{longtable}
\usepackage{graphicx}
\makeatletter
\def\maxwidth{\ifdim\Gin@nat@width>\linewidth\linewidth\else\Gin@nat@width\fi}
\def\maxheight{\ifdim\Gin@nat@height>\textheight\textheight\else\Gin@nat@height\fi}
\makeatother
% Scale images if necessary, so that they will not overflow the page
% margins by default, and it is still possible to overwrite the defaults
% using explicit options in \includegraphics[width, height, ...]{}
\setkeys{Gin}{width=\maxwidth,height=\maxheight,keepaspectratio}
% Set default figure placement to htbp
\makeatletter
\def\fps@figure{htbp}
\makeatother

\KOMAoption{captions}{tableheading}
\makeatletter
\@ifpackageloaded{caption}{}{\usepackage{caption}}
\AtBeginDocument{%
\ifdefined\contentsname
  \renewcommand*\contentsname{Table of contents}
\else
  \newcommand\contentsname{Table of contents}
\fi
\ifdefined\listfigurename
  \renewcommand*\listfigurename{List of Figures}
\else
  \newcommand\listfigurename{List of Figures}
\fi
\ifdefined\listtablename
  \renewcommand*\listtablename{List of Tables}
\else
  \newcommand\listtablename{List of Tables}
\fi
\ifdefined\figurename
  \renewcommand*\figurename{Figure}
\else
  \newcommand\figurename{Figure}
\fi
\ifdefined\tablename
  \renewcommand*\tablename{Table}
\else
  \newcommand\tablename{Table}
\fi
}
\@ifpackageloaded{float}{}{\usepackage{float}}
\floatstyle{ruled}
\@ifundefined{c@chapter}{\newfloat{codelisting}{h}{lop}}{\newfloat{codelisting}{h}{lop}[chapter]}
\floatname{codelisting}{Listing}
\newcommand*\listoflistings{\listof{codelisting}{List of Listings}}
\makeatother
\makeatletter
\makeatother
\makeatletter
\@ifpackageloaded{caption}{}{\usepackage{caption}}
\@ifpackageloaded{subcaption}{}{\usepackage{subcaption}}
\makeatother

\ifLuaTeX
  \usepackage{selnolig}  % disable illegal ligatures
\fi
\usepackage{bookmark}

\IfFileExists{xurl.sty}{\usepackage{xurl}}{} % add URL line breaks if available
\urlstyle{same} % disable monospaced font for URLs
\hypersetup{
  pdftitle={Curriculum Vitae - Rodrigo Hermont Ozon},
  pdfauthor={Rodrigo Hermont Ozon},
  colorlinks=true,
  linkcolor={blue},
  filecolor={Maroon},
  citecolor={Blue},
  urlcolor={Blue},
  pdfcreator={LaTeX via pandoc}}


\title{Curriculum Vitae - Rodrigo Hermont Ozon}
\author{Rodrigo Hermont Ozon}
\date{2025-03-01}

\begin{document}
\maketitle


Rodrigo Hermont Ozon

Economista (Econometria) e Data Scientist

\subsubsection{Contato}\label{contato}

\begin{itemize}
\tightlist
\item
  \textbf{E-mail:}
  \href{mailto:rodrigoozon@yahoo.com.br}{\nolinkurl{rodrigoozon@yahoo.com.br}}
\item
  \textbf{Whatsapp:} +55 (41) 98838-2904
\item
  \textbf{LinkedIn:}
  \href{https://www.linkedin.com/in/rodrigohermontozon/}{rodrigohermontozon}
\item
  \textbf{Google Scholar:}
  \href{https://scholar.google.com/citations?hl=en&user=hPcIR9oAAAAJ}{Perfil}
\item
  \textbf{Portfolio:} \href{https://rhozon.github.io/}{rhozon.github.io}
\end{itemize}

\subsubsection{Habilidades}\label{habilidades}

\begin{itemize}
\tightlist
\item
  Análise de Dados \& Econometria
\item
  Data Science \& Machine Learning
\item
  Market Intelligence \& Precificação
\item
  Planejamento Estratégico e Gerenciamento de Projetos
\end{itemize}

\subsubsection{Competências Técnicas}\label{competuxeancias-tuxe9cnicas}

\begin{itemize}
\tightlist
\item
  \textbf{Linguagens:} R, Python, SQL, VBA, DAX, LaTeX, HTML/CSS
\item
  \textbf{Ferramentas:} Eviews, Stata, Power BI, Tableau, Google Data
  Studio
\end{itemize}

\subsubsection{Idiomas}\label{idiomas}

\begin{itemize}
\item
  \textbf{Português:} Nativo
\item
  \textbf{Inglês:} Avançado
\item
  \textbf{Espanhol:} Intermediário
\end{itemize}

\subsubsection{Perfil Profissional}\label{perfil-profissional}

Econometrista e pesquisador Ph.D.~com experiência em análise de dados,
modelagem e apresentação de informações para apoiar decisões
estratégicas. Atuante nos setores público e privado, com expertise em
soluções digitais e liderança de projetos de inteligência analítica.

\subsubsection{Experiência
Profissional}\label{experiuxeancia-profissional}

\paragraph{Data Engineer Professional \textbar{} VOLVO GROUP DIGITAL \&
IT}\label{data-engineer-professional-volvo-group-digital-it}

{Curitiba, PR -- Jan 2025 a Presente}

\begin{itemize}
\tightlist
\item
  Criação e gerenciamento de pipelines e processos de dados.
\item
  Desenvolvimento de algoritmos customizados e modelos preditivos.
\item
  Colaboração com times de Data Science para otimização de estratégias.
\end{itemize}

\paragraph{Data Scientist Area Coordinator \textbar{}
LAB.PR}\label{data-scientist-area-coordinator-lab.pr}

{Curitiba, PR -- Out 2024 a Dez 2024}

\begin{itemize}
\tightlist
\item
  Coordenação de operações em inteligência pública e integração de
  sistemas.
\item
  Liderança em iniciativas de desenvolvimento tecnológico e parcerias
  institucionais.
\end{itemize}

\paragraph{Data Science Consultant \textbar{} CASE NEW
HOLLAND}\label{data-science-consultant-case-new-holland}

{Curitiba, PR -- Abr 2023 a Out 2024}

\begin{itemize}
\tightlist
\item
  Consultoria em análises preditivas e inferências estatísticas.
\item
  Suporte ao desenvolvimento de modelos de machine learning e
  estratégias de negócio.
\end{itemize}

\emph{Outras experiências relevantes fazem parte da trajetória
profissional.}

\subsubsection{Formação Acadêmica}\label{formauxe7uxe3o-acaduxeamica}

\begin{itemize}
\tightlist
\item
  \textbf{Ph.D.~em Sistemas e Produção} -- PUCPR (2022-2027)
\item
  \textbf{Mestrado em Desenvolvimento Econômico} -- UFPR (2011)
\item
  \textbf{Bacharelado em Economia} -- UFPR (2007)
\item
  \textbf{TechGrad em Data Science} -- Univ. Cruzeiro do Sul (2022)
\item
  \textbf{Licenciatura em Matemática} -- UNESPAR (2002)
\end{itemize}

\subsubsection{Certificações e
Cursos}\label{certificauxe7uxf5es-e-cursos}

\begin{itemize}
\tightlist
\item
  Certificações Datacamp (2024 -- Atual)
\item
  Data Science Professional Certificate -- IBM (2020-2021)
\item
  Econometrics: Methods and Applications -- Erasmus University Rotterdam
  (2020)
\item
  Data Science Math Skills -- Duke University (2020)
\item
  Microsoft Power BI para Data Science (2020)
\end{itemize}

\subsubsection{Publicações \&
Palestras}\label{publicauxe7uxf5es-palestras}

Diversas publicações e palestras em eventos acadêmicos e corporativos.\\
\href{http://lattes.cnpq.br/3532649625879285}{Confira mais detalhes no
Lattes}.




\end{document}
